Advancements in Generative-AI and large language models have increased concerns about academic dishonesty and plagiarism in take-home assignments within computing education. Extensive research has linked key traits from the Big Five personality model to various forms of anti-social behavior. However, there is a notable gap in research regarding the association between personality traits and plagiarism, particularly in programming assignments within computing education. This study addresses this gap by investigating the relationship between the Big Five personality traits and plagiarism scores in programming assignments among undergraduate students. The study was conducted in an artificial intelligence course with 105 participants (N=105) at a large private university in India. Additionally, the study explores whether educating students about academic integrity and clearly defining plagiarism expectations affect plagiarism rates. Our findings suggest that while educating students about plagiarism has a limited impact on reducing malpractice in programming assignments, the extraversion trait of the Big Five personality exhibits a positive association and the conscientiousness trait exhibits a negative association with plagiarism tendencies.