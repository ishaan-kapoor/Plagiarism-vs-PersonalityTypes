
Educating students about academic integrity expectations has been suggested as one of the ways to reduce malpractice in take-home programming assignments. We test this hypothesis using data collected from an artificial intelligence course with 105 participants (N=105) at a university in India. The AI course had two programming assignments. Plagiarism through collusion was quantified using the Measure of Software Similarity (MOSS) tool. Students were educated about what constitutes academic dishonesty and were required to take an honor pledge before the start of the second take-home programming assignment. The two programming assignments were novel and did not have solutions available on the internet. We expected the mean percentage of similar lines of code to be significantly less in the second programming assignment. However, our results show no significant difference in the mean percentage of similar lines of code across the two programming assignments. We also study how the Big-five personality traits affect the propensity for plagiarism in the two take-home assignments. Our results across both assignments show that the extraversion trait of the Big Five personality exhibits a positive association, and the conscientiousness trait exhibits a negative association with plagiarism tendencies. 
Our result suggests that the policy of educating students about academic integrity will have a limited impact as long as students perceive an opportunity for plagiarism to be present. We explain our results using the Fraud triangle model.

% Previous research has linked key traits from the Big Five personality model to various forms of anti-social behavior. However, there is a notable gap in research regarding the association between personality traits and plagiarism, particularly in programming assignments within computing education.

% Advancements in Generative-AI and large language models have increased concerns about academic dishonesty and plagiarism in take-home assignments within computing education. Extensive research has linked key traits from the Big Five personality model to various forms of anti-social behavior. However, there is a notable gap in research regarding the association between personality traits and plagiarism, particularly in programming assignments within computing education. This study addresses this gap by investigating the relationship between the Big Five personality traits and plagiarism scores in programming assignments among undergraduate students. The study was conducted in an artificial intelligence course with 105 participants (N=105) at a large private university in India. Additionally, the study explores whether educating students about academic integrity and clearly defining plagiarism expectations affect plagiarism rates.
% Our findings suggest that educating students about plagiarism does not have a significant effect on plagiarism through collusion.
% Our results also show that extraversion trait of the Big Five personality exhibits a positive association and the conscientiousness trait exhibits a negative association with plagiarism tendencies.
% Our results show that a policy of educating the students about academic integrity will have limited impact on curbing plagiarism. A fraction of students will indulge in plagiarism through collusion as long as they perceive an opportunity to be present.

