% !TeX root = main.tex
\section{Introduction} \label{sec:intro}

Academic integrity is viewed as one of the significant challenges in education. The advancements in Generative AI have led to increased plagiarism in academic assignments \cite{shiriChatGPTAcademicIntegrity2023}. Over the years, a wide range of research has investigated this topic, from exploring why students cheat to developing automatic methods of detecting those who do \cite{10.5120/ijca2015906113}.  

%About Plagiarism in computing courses
According to McCabe et al. \cite{84f6c08b47844b54868caf82c625fc66}, behaviors such as cut-and-paste, which faculty members view as plagiarism, are becoming more acceptable to students. Even when students recognize its wrongness, many admit to engaging in academic dishonesty during their college years\cite{Bernardi2004-BERETD}. Addressing academic integrity within the computing domain poses distinct challenges due to inherent issues within the discipline. The usual rules and policies of the university do not always fit well with computing assignments. For instance, attributing source code according to standard practices may compromise its syntax, while directly quoting external content or code can lead to syntax errors. Despite ongoing efforts, a universally accepted format for attributing externally sourced program code remains elusive \cite{10.1145/1562877.1562900}. The ITiCSE’16 Working Group \cite{10.1145/3024906.3024910} advocated for instructors to articulate their expectations regarding academic integrity in their courses clearly and provided examples of how to convey them. Simon et al.\cite{10.1145/3160489.3160502} subsequently suggested a standardized form for acknowledging the use of source code authored by others. 

In undergraduate programming assignments, code stubs are often provided, and students are encouraged to use examples that increase the suspicion of plagiarism \cite{10.5555/1151869.1151888}. Additionally, in programming assignments, particularly at lower levels, there may be limited ways to express certain features \cite{10.1145/3160489.3160502}, which could worsen this issue.

%About Big 5 personality briefly 
The Big Five personality traits, also known as the Five Factor Model (explained further in the next section), are five broad dimensions that capture different aspects of human personality such as Openness to experience (O), Conscientiousness (C), Extraversion (E), Agreeableness (A), and Neuroticism (N). The influence of the Big Five personality traits is a subject of study across various fields, including its influence on anxiety \cite{AnxityBigfive}, personality development \cite{Tetzner}, and academic achievement \cite{AchievementBigfive, https://doi.org/10.1111/jopy.12663, OZ2015PER}. It has also been studied in the context of academic plagiarism in education \cite{Giluk2015BigFP, Bhutto2019ACS}. However, while previous research has explored the relationship between these traits and the tendency to plagiarize, much of this work focuses on predicting plagiarism propensity rather than directly measuring plagiarism scores and their correlation with personality traits. In our study, we aim to address this gap by examining the actual plagiarism scores alongside the big five trait scores of students, to identify the influence of these traits on plagiarism in programming assignments.

%About how there is confusion among students between what is plagiarism and what is not
While most students clearly recognize behaviors like exam fraud and outsourcing assignments as forms of cheating, there is notable confusion among both students and academics within the computing field regarding the distinctions surrounding plagiarism and collusion in programming assignments \cite{SCPlagiarism}. Numerous studies have highlighted genuine misunderstandings among students regarding what constitutes plagiarism in programming assignments \cite{SCPlagiarism, Cosma2008TowardsAD, 10.1145/637610.544468, 10.1145/2526968.2526971}. This lack of clarity among both faculty and students heightens the risk of unintentional and possibly inadvertent engagement in inappropriate academic practices. While there exist efforts to address acknowledging the use of externally authored source code and defining plagiarism guidelines for students, the effectiveness of these measures remains unassessed. In our study, we explicitly communicate to students the expectations and criteria for plagiarism in programming assignments and obtaining signed honor pledges. Subsequently, we investigate the impact of these interventions on students' behavior regarding plagiarism. In summary, the research questions of this work are as follows:

%Research questions
\begin{itemize}
    \item \label{RQ1} \textbf{RQ1:} What influence do the big-five personality traits have on plagiarism in programming assignments?
    \item \label{RQ2} \textbf{RQ2:} To what extent does sensitizing students about academic integrity and the criteria for plagiarism through an honor pledge influence their behavior?
\end{itemize}

The Sec \ref{sec:frameworks} briefly explains the various theoretical frameworks used in this work and relates to computing education literature in Sec \ref{sec:relatedwork}. The methodology is explained in Sec \ref{sec:method} and the results are presented and discussed in Sec \ref{sec:findings}. The limitations and next steps are highlighted in Sec \ref{sec:limitations} and our findings are concluded in the Sec \ref{sec:conclusion}. 

