% !TeX root = main.tex
\section{Introduction} \label{sec:intro}

Academic integrity, essential for upholding ethical standards in academia, is often gauged by behaviors that defy it—such as seeking and using unauthorized assistance in assignments and exams. It remains a pivotal challenge in education. Computer science(CS) students often face challenges in completing programming assessments, stemming from poor time management and a scarcity of suitable resources \cite{10.5555/858403.858418}. When students cannot access sufficient support, they may resort to cheating as a desperate measure \cite{10.1145/3059009.3059065, 10.1145/1632149.1632168}. Over the years, a wide range of research has investigated this topic, from exploring why students plagiarize to developing automatic methods of detecting those who do \cite{10.5120/ijca2015906113}.

Researchers have identified various forms of academic dishonesty \cite{40db574cf24f441d994cbb1cd909bd1f}, including but not limited to:

\begin{itemize}
    \item Collaborating excessively on take-home assignments,
    \item Copying assignments in part or whole from other students,
    \item Seeking assistance from the Internet to solve challenging problems,
    \item Submitting identical work for multiple courses,
    \item Plagiarizing text from external sources,
    \item Outsourcing the assignment to someone by paying them,
    \item Using undisclosed resources during examinations, among other practices.
\end{itemize} 

Academic dishonesty is classified into three primary categories: cheating, plagiarism, and collusion \cite{8937362}. In this work, we focus on the latter two. While some use the term `plagiarism' to encompass both, there exists a clear distinction between the two in the context of CS disciplines. \textit{Plagiarism} involves using the work of `others' without proper attribution, typically sourced from public mediums like GitHub, journals, or the internet. On the other hand, \textit{collusion} involves collaborating with `known' individuals (typically classmates) to produce academic work without proper attribution. An example of plagiarism would be incorporating unattributed content from the web, while collusion involves students collaborating on individual assignments meant to be completed independently \cite{Owunwanne, Fraser2014CollaborationCA,JonesJuliet}.

While much of the existing research on academic integrity focuses on textual plagiarism, there is growing recognition that plagiarism can occur across various non-text mediums such as source code and multimedia \cite{Marcin2016,10.1145/2526968.2526971}.

%About Plagiarism in computing courses
According to McCabe et al.\cite{84f6c08b47844b54868caf82c625fc66}, behaviors such as cut-and-paste, which faculty members view as plagiarism, are becoming more acceptable to students. Even when students recognize its wrongness, many admit to engaging in academic dishonesty during their college years\cite{Bernardi2004-BERETD}. Addressing academic integrity within the computing domain poses distinct challenges due to inherent issues within the discipline. The usual rules and policies of the university do not always fit well with programming assignments. For instance, attributing source code according to standard practices such as quoting the original work or citing them may compromise the syntax of the code and can lead to syntax errors. 
In undergraduate programming assignments, code stubs are often provided, and students are encouraged to use examples that increase the suspicion of plagiarism \cite{10.5555/1151869.1151888}. Additionally, in programming assignments, particularly at lower levels such as assembly language, there may be limited ways to express certain features \cite{10.1145/3160489.3160502}, which could worsen this issue.

Despite ongoing efforts, a universally accepted format for attributing externally sourced program code remains elusive \cite{10.1145/1562877.1562900}. The ITiCSE’16 Working Group \cite{10.1145/3024906.3024910} advocated for instructors to articulate their expectations regarding academic integrity in their courses clearly and provided examples of how to convey them. Simon et al.\cite{10.1145/3160489.3160502} subsequently suggested a standardized form for acknowledging the use of source code authored by others. 

%About how there is confusion among students between what is plagiarism and what is not
While most students clearly recognize behaviors like exam fraud and outsourcing assignments as forms of cheating, there is notable confusion among both students and academics within the computing field regarding the distinctions surrounding plagiarism and collusion in programming assignments \cite{SCPlagiarism}. Numerous studies have highlighted genuine misunderstandings among students regarding what constitutes plagiarism in programming assignments \cite{SCPlagiarism, Cosma2008TowardsAD, 10.1145/637610.544468, 10.1145/2526968.2526971}. This lack of clarity among both faculty and students heightens the risk of unintentional and possibly inadvertent engagement in inappropriate academic practices. 

Multiple researchers suggest educating students about plagiarism (both coincidental and non-coincidental), and establishing clear academic integrity standards as a possible approach to mitigating instances of plagiarism and collusion in take-home programming assignments\cite{10.1145/3506717, 10.1145/2591708.2591755, 10.1145/2632320.2632342, Joy2011SourceCP}. However, this hypothesis has yet to be tested in a developing country like India. Given the substantial enrollment of over 2 million undergraduates enrolled in computer science and related courses across Indian higher education institutes \cite{departmentofhighereducationgovernmentofindiaAllIndiaSurvey2021}, along with raising concerns on student plagiarism within the country\cite{bakthavatchaalam2021academic} and among Indian students studying outside India\cite{plagiairmInIndia}, it becomes crucial to evaluate the efficacy of these strategies in varied cultural and geographical landscapes.

In our study, we clearly outline to students the guidelines regarding plagiarism and collusion in programming tasks, along with obtaining signed honor pledges. We then examine how these measures affect students' conduct concerning plagiarism and collusion.

%About Big 5 personality briefly 
On a parallel trajectory, the Big Five personality traits, also known as the Five Factor model or OCEAN model (explained further in the next section), are five broad dimensions that capture different aspects of human personality such as Openness to experience (O), Conscientiousness (C), Extraversion (E), Agreeableness (A), and Neuroticism (N). The influence of the Big Five personality traits is a subject of study across various fields, including its influence on anxiety \cite{AnxityBigfive}, personality development \cite{Tetzner}, and academic achievement \cite{AchievementBigfive, https://doi.org/10.1111/jopy.12663, OZ2015PER}. 

It has also been studied in the context of academic plagiarism in education \cite{Giluk2015BigFP, Bhutto2019ACS}. However, while previous research has explored the relationship between these traits and the tendency to plagiarize, much of this work focuses on predicting plagiarism propensity rather than directly measuring plagiarism scores and their correlation with personality traits. In our study, we aim to address this gap by examining the actual plagiarism scores alongside the big five trait scores of students to identify the influence of these traits on plagiarism in programming assignments. In summary, the research questions of this work are as follows:

\begin{itemize}
    \item \label{RQ1} \textbf{RQ1:} What influence do the big-five personality traits have on plagiarism in programming assignments?
    \item \label{RQ2} \textbf{RQ2:} To what extent does sensitizing students about academic integrity and the criteria for plagiarism and collusion through an honor pledge influence their behavior?
\end{itemize}

We approached RQ1 as exploratory, without a specific hypothesis. For RQ2, we anticipated a decrease in the mean percentage of similar lines of code in the second programming assignment following student sensitization through an honor pledge. This study, to our knowledge, is the first of its kind, and its findings could be valuable for the computing education community. 

The Sec \ref{sec:frameworks} briefly explains the various theoretical frameworks used in this work and relates to computing education literature in Sec \ref{sec:relatedwork}. The methodology is explained in Sec \ref{sec:method}. The results are presented in Sec \ref{sec:findings} and discussed in \ref{sec:discuss}. The limitations and next steps are highlighted in Sec \ref{sec:limitations} and our findings are concluded in Sec \ref{sec:conclusion}. 
