% !TeX root = main.tex
\section{Limitations and Future Work}
\label{sec:limitations}
This study has several limitations that should be noted. Firstly, the sample size was relatively small, consisting of only 105 students from a single institution in India, which may limit the extent to which the findings can be generalized to broader populations. Additionally, the cross-sectional design of the study hinders our ability to establish causal relationships between personality traits and plagiarism behavior. Furthermore, the study primarily relied on quantitative analysis, overlooking the nuanced qualitative aspects of students' perceptions and experiences regarding plagiarism and collusion which could bring about contextual factors, such as cultural norms and academic pressures, which may influence plagiarism behavior.

Future research endeavors could address these limitations by incorporating qualitative interviews to gain deeper insights into students' attitudes, motivations, and plagiarism-related experiences. Longitudinal studies could provide a more comprehensive understanding of the dynamics between personality traits and plagiarism behavior over time, particularly following interventions aimed at sensitizing students about plagiarism. Additionally, replicating the study in diverse educational settings and cultural contexts could enhance the robustness of the findings and improve generalizability.
 