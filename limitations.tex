% !TeX root = main.tex
\section{Limitations and Future Work}
\label{sec:limitations}
This study has several limitations that should be noted. As a result, the findings may not be directly applicable to other educational environments or programming tasks that differ significantly in scope and complexity.

\begin{itemize}
    \item \textit{Single Institution Context:} Our study was conducted in a single institution in India, which may limit the generalizability of our findings. Different educational settings and cultures may influence the effectiveness of the honor pledge intervention.
    \item \textit{Sample Size and Demographics:} The relatively small sample size of 105 students may not capture the diversity of student behavior and attitudes toward plagiarism. Larger, more diverse samples are needed to validate our findings.
    \item \textit{Influence of Pre-existing Academic Integrity Culture:} The existing academic integrity culture at the institution may have influenced the outcomes. Institutions with different levels of emphasis on academic integrity might experience different results from the intervention.
    \item \textit{Impact of External Factors:} As discussed in the discussion, external factors, such as peer influence, \textit{pressure} to excel, perceived \textit{opportunity}, and the difficulty of the assignments, could have contributed to the outcomes. These factors were not controlled for in the study and may have influenced the results.
    \item \textit{Comparison of Different Types of Assignments:} Our study focused on two programming assignments, which may not be representative of all types of assignments. The nature of the assignments and the presence of code stubs or templates could affect the incidence of plagiarism and the effectiveness of the honor pledge. We tried to reduce the severity of this issue by using the -b parameter while executing MOSS as discussed in Sec \ref{sec:method}.
    \item \textit{Absence of Control Group:} We could have designed the experiment with two groups – a control group and an honor pledge group – for each of the two programming assignments. However, we felt that the results would be difficult to interpret if collusion occurred between students of different groups. For example, if a student from the control group colluded with a student from the honor pledge group, it would complicate the interpretation of the results. Therefore, we avoided using separate groups and the current study focuses only on comparing the percentage of code similarity across two different programming assignments. Although we avoided using separate control and honor pledge groups to prevent collusion and interpretational difficulties, this decision introduces a potential confounding variable. Future studies with a clear control group could provide more robust evidence of the honor pledge's effectiveness.
    \item \textit{Random Chance and Personality Traits:} The cross-sectional design of the study limits our ability to establish causal relationships between personality traits and plagiarism behavior. It is possible that, due to random chance, participants who engaged in plagiarism in both programming assignments had higher scores for the extraversion personality trait. This could result in a positive correlation between extraversion and the percentage of code similarity across both programming assignments.
    \item \textit{Gender Distribution and Cultural Homogeneity:} In India, engineering programs exhinit a severe imbalance in gender represention. For example, at our institution, only 12.5\% of the undergraduate students in the computer science program identify as female. The same is reflected in this study with only 6.6\% of participants being female. This disproportionate gender distribution (98 men to 7 women) and cultural homogeneity (all participants from India) may limit the generalizability of the study's findings to other demographic groups or cultural contexts.  
    \item \textit{Personality Test Timing:} Administering the personality test after the second assignment (at the end of the semester) might not accurately reflect the personality traits, as experiences from the course or external factors could alter trait expressions \emph{temporarily}.
\end{itemize}

Furthermore, the study primarily relied on quantitative analysis, overlooking the nuanced qualitative aspects of students' perceptions and experiences regarding plagiarism and collusion which could bring about contextual factors, such as cultural norms and academic pressures, which may influence plagiarism behavior.

Future research endeavors could address these limitations by incorporating qualitative interviews to gain deeper insights into students' attitudes, motivations, and plagiarism-related experiences. Longitudinal studies could provide a more comprehensive understanding of the dynamics between personality traits and plagiarism behavior over time, particularly following interventions aimed at sensitizing students about plagiarism. Additionally, replicating the study in diverse educational settings and cultural contexts could enhance the robustness of the findings and improve generalizability.
 